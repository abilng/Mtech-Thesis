\appendix
\label{app:pythondnn}
\chapter{How to Use : Python-DNN}
%\label{app:pythondnn}
\section{Installation}
\subsection{As Toolkit}

\noindent \textit{Python-DNN} is hosted publicly in github. One can get latest version by cloning git repository.
\begin{lstlisting}[language=bash,basicstyle=\small] 
$ git clone git://github.com/IITM-DONLAB/python-dnn.git
\end{lstlisting}
Or one can get tarball from \\
\url{https://github.com/IITM-DONLAB/python-dnn/tarball/master}

\subsubsection{Dependencies}
\begin{itemize}
	\item \href{https://www.python.org/downloads/}{Python} ($\geq2.6$, $< 3.0$),
	\item \href{http://www.numpy.org/}{NumPy} ($\geq 1.6.1$),
	\item \href{http://www.deeplearning.net/software/theano/install.html#install}{Theano} ($\geq 0.7$)
	\item \href{http://matplotlib.org/}{matplotlib} ($\geq 1.4.3$).
\end{itemize}

\subsection{As Library}
\textit{Python-DNN} can be installed as a library via 
\textit{pip or easy\_install}.
\begin{lstlisting}[language=bash,basicstyle=\small] 
# pip install https://github.com/IITM-DONLAB/python-dnn/zipball/master
\end{lstlisting}

\section{Toolkit Configuration} 
All configation files are in json format.
\subsection{Model Configuration (model.conf)}
\begin{table}[h]
   \begin{center}
   \begin{tabular}{|l|p{10cm}|l|} \hline
   	\textbf{parameter} & \textbf{description} & \textbf{default}\\  \hline
     *nnetType & type of Network (CNN/RBM/SDA/DNN) & \\
     *train\_data & working directory containing data configuration and output \\ \hline
	 *wdir & working directory & \\ \hline
 	 *data\_spec & path for data sepification relative to model.conf & \\ \hline
	 *nnet\_spec & path for network configuration specification relative to model.conf & \\ \hline
	 *output\_file & path for network model file relative to wdir & \\ \hline
	 input\_file & path for pre-trained/fine-tuned model relative to wdir. & \\ \hline
	 random\_seed & random seed for initialization of weights & none \\ \hline
	 logger\_level & logger level : `INFO',`DEBUG' and `ERROR' & `INFO' \\ \hline
	 batch\_size & mini batch size  & 128 \\ \hline
	 *n\_ins & input dimension  & \\ \hline
	 *n\_outs & output dimension (num classes) & \\ \hline
	 finetune\_params& Refer \nameref{subsec:finetuneparam}	&  \\ \hline
	 pretrain\_params& Refer \nameref{subsec:pretrainparam}	&  \\ \hline
	 export\_path &  path for writting (bottleneck) features relative to model.conf & \\ \hline 	 
	 processes & 
	 	\begin{tabular}{r|p{6cm}} %\hlne
	 	 pretraining & to do pre-training or not \\
	 	 finetuning & to do fine-tuning or not \\
	 	 testing & to do testing or not \\
	 	 export\_data & to do feat extraction or not. If true export\_path is mandatory \\
 	   \end{tabular}	
	 & false \\ \hline
	 save\_freq & epoch interval for saving model & \\ \hline 	 	
   \end{tabular}		
   \end{center}
 \end{table} 

\subsubsection{Pretraining Parameters}
\label{subsec:pretrainparam}
\emph{pretrain\_params}: Configuration of pretraining method. It contains a json object with following parameters:
\begin{table}[h]
\centering
\begin{tabular}{|l|l|l|p{6cm}|}
\hline
 Param	 & Default value	 & nnet Type	 & Description\\
\hline
 \emph{gbrbm\_learning\_rate}	 &     0.005	 &    DBN	 & Pretraining learning rate for gbrbm layer.\\
 \emph{learning\_rate}	 &      0.08	 &  SDA,DBN	 & Pretraining learning rate (DBN: for all layers except gbrbm layer)\\
 \emph{epochs}	 &       15	 &    DBN	 & No of Pretraining epochs\\
 \emph{initial\_momentum}	 &      0.5	 &    DBN	 & The initial momentum factor while pre-training\\
 \emph{final\_momentum}	 &      0.9	 &    DBN	 & The final momentum factor while pre-training\\
 \emph{initial\_momentum\_epoch}	 &       5	 &    DBN	 & No: of epochs with the initial momentum factor before switching\\
 \emph{keep\_layer\_num}	 &       0	 &  SDA,DBN	 & From which layer Pre-Trainig Should Start.\\
 \hline
\end{tabular}
\end{table}

\subsubsection{Finetune Parameters}
\label{subsec:finetuneparam}
\emph{finetune\_params} contain configuration of fine-tune learning method.There are 2 learning rate adaption technique supported by toolkit. They are
\begin{itemize}
\item C: Constant learning rate: run `epoch\_num' iterations with `learning\_rate' unchanged
\item E: Exponential decay: we start with the learning rate of \emph{start\_rate}; if the validation error reduction between two epochs is less than \emph{min\_derror\_decay\_start}, the learning rate is scaled by \emph{scale\_by} during each of the remaining epoch. The whole traing terminates when the validation error reduction between two epochs falls below \emph{min\_derror\_stop}. \emph{min\_epoch\_decay\_star} is the minimum epoch number after which scaling can only be performed.
\end{itemize}
\begin{table}[h]
\centering
\begin{tabular}{|l|l|l|}
\hline
\textbf{parameter}	& \textbf{description} 				& \textbf{default}\\  \hline
momentum 			& momentum factor for finetuning 	& \\
method 				& E : exponential decay 			& C\\
					& C : constant learning rate 		& \\
learning\_rate   	& used for C 						& 0.08\\
epoch\_num          & used for C 			 			& 10\\
start\_rate         & used for E 						& 0.08 \\
scale\_by           & used for E 						& 0.5\\
min\_derror\_decay\_start& used for E 					& 0.05\\
min\_derror\_stop   & used for E 						& 0.05 \\
min\_epoch\_decay\_start  & used for E 					& 15\\
\hline
\end{tabular}
\caption[]{finetune\_params' contains a json object with these parameters}
\end{table}

\clearpage

\subsection{Data Configuration (data.conf)}
\begin{table}[h]
\begin{center}
  \medskip  \small configuration for training/testing/validation.
   \begin{tabular}{|l|p{8cm}|c|} \hline
   	\textbf{parameter} & \textbf{description} & \textbf{default}\\  \hline
 	*base\_path & Base path of data. &  \\  \hline
   	*filename &  train/test/val filename & \\  \hline
	*partition & data size (in MiB) to be loaded in memory & \\  \hline
	random & allow random ordering  & true \\  \hline
	random\_seed & seed for random numbers if random is true & \\  \hline 
	keep\_flatten & flatten vector or reshape to input\_shape & false \\  \hline
	*reader\_type & reader type : NP/T1/T2. & \\  \hline		
	*input\_shape & shape of input data & \\  \hline
	*dim\_shuffle &  shuffle order of the input data &  \\ \hline
  \end{tabular}		
\end{center}
\end{table} 
\noindent For the purpose of using the toolkit data have to be in anyone of the file format.
\begin{itemize}
\item{\textbf {Numpy Format [NP]:}The dataset is stored as single file in binary format in following structur,
\begin{lstlisting}[language=bash,basicstyle=\small] 
<json-header>
<structured numpy.array>
<structured numpy.array>
..
\end{lstlisting}
json-header  : featdim (dimension of input vector after flattening), input\_shape (shape of input).}

\item{\textbf {Text File (One level header) [T1]:} The dataset contains a root file with list of  text file names corresponding to a class. It has following format,
\begin{lstlisting}[language=bash,basicstyle=\small] 
<feat_dim> <num_classes>
<data_file1>
<data_file2>
..
\end{lstlisting}
data\_file correspond to individual class files with following structure,
\begin{lstlisting}[basicstyle=\small] 
<feat_dim> <num_feat_vectors(optional)>
<feat_vector>
<feat_vector>
..
\end{lstlisting}}

\item{\textbf {Text File (Two level header) [T2]:} This format has got extra level of indirection, the root file has got following structure,
\begin{lstlisting}[language=bash,basicstyle=\small] 
<feat_dim> <num_classes>
<class_index_file1>
<class_index_file2>
..
\end{lstlisting}
class\_index\_file constitute of list of data filenames belonging to single class, 
\begin{lstlisting}[basicstyle=\small] 
<data_file1>
<data_file2>
..
\end{lstlisting}}
\end{itemize}

\subsection{Network Configuration (nnet.conf)}
\begin{table}[!htbp]
\begin{center}
  \medskip  \small configuration of cnn
   \begin{tabular}{|c|p{12cm}|} \hline
   	\textbf{parameter} & \textbf{description} \\  \hline
   	 cnn & 
	 \begin{tabular}{c|p{9cm}} %\hlne
	 layers+ & 
		\begin{tabular}{r|p{6cm}} %\hlne
		convmat\_dim & dimension of convolution weights \\  \hline
		num\_filters & number of feature maps \\  \hline
		poolsize & max-pooling dimensions \\  \hline
		update & updated weight during training. \\  \hline
		activation & activation function used by the layer \\ 
		\end{tabular} \\ \hline
	  activation & global activation function \\ \hline
	  use\_fast & use pylearn2 library for faster computation \\ 
 	   \end{tabular}	 \\ \hline
 	 mlp & 
	 \begin{tabular}{c|p{9cm}} %\hlne 
	  layers &  hidden layer sizes \\ \hline
	  adv\_activation & 
		\begin{tabular}{r|p{6cm}} %\hlne
			method &  'maxout','pnorm' \\ \hline
			pool\_size & pool size (in pnorm) \\ \hline
			pnorm\_order & norm order for pnorm. (Default: 1) \\
		\end{tabular} \\ \hline
	  activation & activation function for mlp layers. (if adv\_activation is used, then either 'linear','relu' or 'cappedrelu') \\ 
 \end{tabular}	 \\ \hline
  \end{tabular}		
\end{center}
 \end{table} 

\begin{table}[!htbp] 
 \begin{center}
  	\medskip  \small configuration of dnn
   	\begin{tabular}{|c|p{8cm}|c|} \hline
   	\textbf{parameter} & \textbf{description} & \textbf{default}\\  \hline
	*hidden\_layers &  RBM layer sizes. & \\ \hline
	pretrained\_layers & number of layers  pre-trained. & 0 \\ \hline
	activation & activation function for the layers. (if adv\_activation is used, then either 'linear','relu' or 'cappedrelu') & tanh or linear \\ \hline
	max\_col\_norm & The max value of norm of gradients (in dropout and maxout)	& null \\ \hline
	l1\_reg &  l1 norm regularization & 0 \\ \hline
	l2\_reg &  l2 norm regularization & 0 \\ \hline
	adv\_activation & 
		\begin{tabular}{r|p{5cm}} %\hlne
		method &  'maxout','pnorm' \\ \hline
		pool\_size & pool size (in pnorm) \\ \hline
		pnorm\_order & norm order for pnorm. (Default: 1) \\
		\end{tabular} & \\ \hline
	do\_dropout &  use dropout or not & false  \\ \hline
	dropout\_factor & dropout factors for DNN layers. & [0.0] \\ \hline
	input\_dropout\_factor & dropout factor for input features & 0.0 \\ \hline
	\end{tabular}
\end{center}
\end{table} 
 
\begin{table}[!htbp] 
 \begin{center}
  	\medskip  \small configuration of dbn (rbm)
	\begin{tabular}{|c|p{8cm}|c|} \hline
   	\textbf{parameter} & \textbf{description} & \textbf{default}\\  \hline
	*hidden\_layers &  RBM layer sizes. & \\ \hline
	activation & activation function for the layers. & tanh \\ \hline
	pretrained\_layers & number of layers  pre-trained. & 0 \\ \hline
	first\_layer\_type & 'bb' (Bernoulli-Bernoulli) or 'gb' (Gaussian-Bernoulli) & gb  \\ 	\hline 
	\end{tabular}		
\end{center}
\end{table} 
\begin{table}[!htbp] 
 \begin{center}
  	\medskip  \small configuration of dbn (sda)
	\begin{tabular}{|c|p{8cm}|c|} \hline
   	\textbf{parameter} & \textbf{description} & \textbf{default}\\  \hline
	*hidden\_layers &  hidden denoising autoencoder layer sizes & \\ \hline
	activation & activation function for the layers & tanh \\ \hline
	*corruption\_levels & corruption level for each layer &  \\ \hline
	\end{tabular}		
\end{center} 
\end{table} 
\noindent It is quite evident in all model configuration the activation functions are crucial. The activation function are used to transform the activation level of a unit (neuron) into an output signal. Generally, activation functions have a "squashing" effect. Python-DNN currently support following activation functions,
\begin{itemize}
\item {\textbf{sigmoid:} sigmoid function with equation: $f(x) = \frac{1}{(1 + \exp^{-x}}$. This is an S-shaped (sigmoid) curve, with output in the range $(0,1)$.}
\item {\textbf{tanh:} hyperbolic tangent function is a sigmoid curve, like the logistic function, except that output lies in the range $(-1,+1)$.} 
\item {\textbf{relu:} rectifier linear unit is an activation function defined as $f(x) = max(0, x)$}
\item {\textbf{cappedrelu:} same as ReLU except we cap the units at 6.ie, $f(x) = min(max(x,0),6)$}
\end{itemize}
.\\
Toolkit contain sample configuration (of MNIST and CIFAR dataset) in \textit{sample\_config/} folder
\clearpage

\section{Usage}

\begin{enumerate}
	\item Prepare Dataset: convert dataset to NP/T1/T2 formats.
	\item Configure model: Note that CNN accepting an image with dimension $X \times Y$ with $K$ channel should have \textit{input\_shape} as $[K,X,Y]$
	\item Configure data: Set the \textit{base\_path} and \textit{filename}. The parameter \textit{partition} need to be set properly in case of large dataset, assign value (in MB) less than primary memory/gpu memory. 
	\item Configure neural net: nnet\_conf.json
	\item Set environment flags: set \textit{device} to either cpu or gpu mode.
	\begin{lstlisting}[language=bash,basicstyle=\small] 
		export THEANO_FLAGS=device=cpu,floatX=float32
	\end{lstlisting}
	\item Run toolkit: The toolkit can be run by executing
	\begin{lstlisting}[language=bash,basicstyle=\small] 
		./python-dnn <model.conf>
	\end{lstlisting}
\end{enumerate}
