\chapter{CONCLUSION AND FUTURE WORK} 
\label{chap:concl}
%%%%%%%%%%%%%%%%%%%%%%%%%%%%%%%%%%%%%%%%%%%%%%%%%%%%%%%%%%%%%%%%%%%%%%%%%%%%%%%%%%%%%%%%%%%%%%%%%%%%%%%%%%%%%%%%%%%%%%%%%
\section{Raw Data}
No one tells a child how to see, especially in the early years. They learn this through real-world experiences and examples. If you consider a child's eyes as a pair of biological cameras, they take one picture about every 200 milliseconds, the average time an eye movement is made. So by age three, a child would have seen hundreds of millions of pictures of the real world. That's a lot of training examples. So instead of focusing solely on better and better algorithms, my insight was to give the algorithms the kind of training data that a child was given through experiences in both quantity and quality.


Little by little, we're giving sight to the machines. First, we teach them to see. Then, they help us to see better. For the first time, human eyes won't be the only ones pondering and exploring our world. We will not only use the machines for their intelligence, we will also collaborate with them in ways that we cannot even imagine. 
%%%%%%%%%%%%%%%%%%%%%%%%%%%%%%%%%%%%%%%%%%%%%%%%%%%%%%%%%%%%%%%%%%%%%%%%%%%%%%%%%%%%%%%%%%%%%%%%%%%%%%%%%%%%%%%%%%%%%%%%
