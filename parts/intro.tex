% Introduction.
\chapter{INTRODUCTION}
\label{chap:intro}

\section{Raw TODO}

\paragraph{Keyword spotting for video soundtrack indexing}
The amount of available video information is dramatically increasing due to the development of multimedia applications. As a consequence, content based retrieval tools are urgently needed for fast and easy access to multimedia database but also to movies and recorded video news. In particular, queries may rely on off-line indexing. Keyword spotting on video soundtracks could be of great help in this indexation process and in the future associated with pattern or event recognition out of the strictly visual information. Specific constraints for this application are identified and a solution based on phonemic lattices is proposed. The word spotter achieves indexing on open vocabularies uttered by any speaker. It is fast enough for practical applications and does not require much additional stored information

\paragraph{Event based indexing of broadcasted sports video by intermodal collaboration}
In this paper, we propose event-based video indexing, which is a kind of indexing by its semantical contents. Because video data is composed of multimodal information streams such as visual, auditory, and textual [closed caption (CC)] streams, we introduce a strategy of intermodal collaboration, i.e., collaborative processing taking account of the semantical dependency between these streams. Its aim is to improve the reliability and efficiency in contents analysis of video. Focusing here on temporal correspondence between visual and CC streams, the proposed method attempts to seek for time spans in which events are likely to take place through extraction of keywords from the CC stream and then to index shots in the visual stream. The experimental results for broadcasted sports video of American football games indicate that intermodal collaboration is effective for video indexing by the events such as touchdown (TD) and field goal (FG)

\paragraph{Temporal Sequence Modeling for Video Event Detection} 
The exponential growth of video content today creates a great need for methods of intelligent video analysis and understanding. Among them, video event detection plays a central role in many applications such as surveillance, topic discovery and content retrieval.   The task  of event detection involves identifying the temporal range of an event in a video (i.e.when) and sometimes the location of the event as well (i.e. where).  While there have been increasing efforts recently to tackle this problem, it remains rather challenging due to compounding issues such as large intra-variances of  events,  varied  durations of  events  and  the  presence  of background clutter